\documentclass{article}

\topmargin 0.0cm
\oddsidemargin 0.2cm
\textwidth 16cm 
\textheight 21cm
\footskip 1.0cm

\renewcommand{\baselinestretch}{0.7} 

\newenvironment{sciabstract}{%
\begin{quote} \bf}
{\end{quote}}


\renewcommand\refname{References and Notes}

\newcounter{lastnote}
\newenvironment{scilastnote}{%
\setcounter{lastnote}{\value{enumiv}}%
\addtocounter{lastnote}{+1}%
\begin{list}%
{\arabic{lastnote}.}
{\setlength{\leftmargin}{.22in}}
{\setlength{\labelsep}{.5em}}}
{\end{list}}


\title{Optimizing School Schedule Generation: A Mathematical Modeling Approach} 

\author
{Ivan Rodin$^{1\ast}$\\
\\
\normalsize{$^{1}$Mathematical optimization developer}\\
\\
\normalsize{$^\ast$To whom correspondence should be addressed; E-mail:  ???.}
}

\date{???}



%%%%%%%%%%%%%%%%% END OF PREAMBLE %%%%%%%%%%%%%%%%



\begin{document} 

\baselineskip24pt

\maketitle 


\begin{sciabstract}
  Scheduling classes and teachers is a challenging problem for schools. With growing student enrollment, there is a need for efficient scheduling practices that meet the school's constraints and requirements. This paper presents a mathematical optimization model for generating optimal schedules using linear and nonlinear programming techniques. The model is implemented using the Pyomo library and solved using optimization solvers such as Gurobi, CPLEX, or HIGHS. The objective of the project is to automate the scheduling process, reducing the time and effort required to generate schedules and minimizing the risk of errors. The results of the optimization model are discussed, highlighting the potential benefits of the proposed solution. This project makes a valuable contribution to the field of scheduling and optimization and serves as a useful reference for schools and researchers.
\end{sciabstract}


\section*{Introduction}

The modern education system is undergoing significant changes with the advent of digitalization. Schools are implementing electronic journals, schedules, books, and holding remote parent meetings, while also creating unified accounting systems to evaluate schools efficiency. However, this digitization also presents new opportunities for students, as they are now able to customize their educational programs and choose subjects and activities that interest them the most through a flexible schedule.

The challenge in creating a school schedule lies in the fact that with each student having a unique program, limited resources such as teachers and classrooms must be used efficiently to meet their educational needs and schedules. The school schedule must also take into account the teacher's working hours, classroom capacity, and preferences for conducting classes, as well as the students' interests and needs.

Thus, the issue of composing a school schedule has become highly relevant in our present time, as schools strive to balance the benefits of digitization with the practicalities of managing limited resources. This research paper aims to explore this issue in depth, and provide possible solutions to help schools create an optimized and efficient schedule that meets the needs of both teachers and students.

\textbf{!!! Add info about LP formulation}

\textbf{!!! Add info about solvers}

\section{Problem formulation}

We will now define the main sets used in the model.

\begin{itemize}
  \setlength\itemsep{0.05em}
    \item $STUDENTS$ -- the set of students;
    \item $TEACHERS$ -- the set of teachers;
    \item $COURSES$ -- the set of courses;
    \item $ROOMS$ -- the set of classrooms;
    \item $GROUPS$ -- the set of groups;
    \item $DAYS$ -- the set of days;
    \item $INTERVALS$ -- the set of intervals, which can be an hour, 45 minutes, or any other time frame.
\end{itemize}

Each element in the sets introduced above represents a class. Each class has certain properties, which we will refer to as model parameters.

\begin{itemize}
    \setlength\itemsep{0.05em}
    \item For each student $s$ in set $STUDENTS$, the following properties are defined:
    \begin{itemize}
        \setlength\itemsep{0.05em}
        \item $P \subseteq COURSES$ -- the set of courses that the student needs to learn (student program);
        \item $g \in GROUPS$ -- the student's group.
    \end{itemize}

    \item For each teacher $t$ in set $TEACHERS$, the following properties are defined:
    \begin{itemize}
        \setlength\itemsep{0.05em}
        \item $Q \subseteq COURSES$ -- the set of courses in which the teacher has the necessary qualifications;
        \item $D \subseteq DAYS$ -- the set of available days;
    \end{itemize}
    
    \item For each classroom $r$ in set $ROOMS$, the following properties are defined:
    \begin{itemize}
        \setlength\itemsep{0.05em}
        \item $A \subseteq COURSES$ -- the set of courses that can be taught in the classroom;
        \item $capacity$ -- the maximum count of students in the classroom;
    \end{itemize}

    \item For each day $d$ in set $DAYS$, the following properties are defined:
    \begin{itemize}
        \setlength\itemsep{0.05em}
        \item $I \subseteq INTERVALS$ -- the set of intervals that are available on the day.
    \end{itemize}

\end{itemize}

\section{Baseline approach}


\bibliography{scibib}

\bibliographystyle{Science}


\end{document}
